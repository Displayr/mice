\HeaderA{mice.impute.logreg}{Elementary Imputation Method: Logistic Regression}{mice.impute.logreg}
\keyword{misc}{mice.impute.logreg}
\begin{Description}\relax
Imputes univariate missing data using logistic regression.
\end{Description}
\begin{Usage}
\begin{verbatim}
    mice.impute.logreg(y, ry, x)
\end{verbatim}
\end{Usage}
\begin{Arguments}
\begin{ldescription}
\item[\code{y}] Incomplete data vector of length n
\item[\code{ry}] Vector of missing data pattern of length n (FALSE=missing,
TRUE=observed) 
\item[\code{x}] Matrix (n x p) of complete covariates.
\end{ldescription}
\end{Arguments}
\begin{Details}\relax
Imputation for binary response variables by the Bayesian 
logistic regression model. See Rubin (1987, p. 169-170) for
a description of the method.
The method consists of the following steps:
\Enumerate{
\item Fit a logit, and find (bhat, V(bhat))
\item Draw BETA from N(bhat, V(bhat))
\item Compute predicted scores for m.d., i.e. logit-1(X BETA)
\item Compare the score to a random (0,1) deviate, and mice.impute.}
The method relies on the standard glm.fit function.
\end{Details}
\begin{Value}
\begin{ldescription}
\item[\code{imp}] A vector of length nmis with imputations (0 or 1).
\end{ldescription}
\end{Value}
\begin{Note}\relax
An alternative is mice.impute.logreg2.
\end{Note}
\begin{Author}\relax
Stef van Buuren, Karin Oudshoorn, 2000
\end{Author}
\begin{References}\relax
Van Buuren, S. \& Oudshoorn, C.G.M. (2000). Multivariate Imputation by Chained Equations: 
MICE V1.0 User's manual. Report PG/VGZ/00.038, TNO Prevention and Health, Leiden.

Brand, J.P.L. (1999). Development, Implementation and Evaluation of Multiple Imputation Strategies for 
the Statistical Analysis of Incomplete Data Sets. Ph.D. Thesis, 
TNO Prevention and Health/Erasmus University Rotterdam. ISBN 90-74479-08-1.
\end{References}
\begin{SeeAlso}\relax
\code{\LinkA{mice}{mice}}, \code{\LinkA{glm}{glm}}, \code{\LinkA{glm.fit}{glm.fit}},
\code{\LinkA{mice.impute.logreg2}{mice.impute.logreg2}}
\end{SeeAlso}

