\HeaderA{mice.mids}{Multivariate Imputation by Chained Equations (Iteration Step)}{mice.mids}
\keyword{misc}{mice.mids}
\begin{Description}\relax
Takes a "mids"-object, and produces an new object of class "mids".
\end{Description}
\begin{Usage}
\begin{verbatim}
mice.mids(obj, maxit=1, diagnostics=TRUE, printFlag=TRUE)
\end{verbatim}
\end{Usage}
\begin{Arguments}
\begin{ldescription}
\item[\code{obj}] An object of class "mids", typically produces by a previous call
to \code{mice()} or \code{mice.mids()} 
\item[\code{maxit}] The number of additional Gibbs sampling iterations. 
\item[\code{diagnostics}] A Boolean flag. If TRUE, diagnostic information will be appended to 
the value of the function. If FALSE, only the imputed data are saved. 
The default is TRUE.
\item[\code{printFlag}] A Boolean flag. If TRUE, diagnostic information during the Gibbs sampling
iterations will be written to the command window.  The default is TRUE.
\end{ldescription}
\end{Arguments}
\begin{Details}\relax
This function enables the user to split up the computations of the 
Gibbs sampler into smaller parts. This is useful for the following
reasons:
\begin{itemize}
\item RAM memory may become easily exhausted if the number of iterations is 
large. Returning to prompt/session level may alleviate these problems.
\item The user can compute customized convergence statistics at specific
points, e.g. after each iteration, for monitoring convergence.
- For computing a 'few extra iterations'.
\end{itemize}
Note: The imputation model itself is specified in the mice() function
and cannot be changed with mice.mids.
The state of the random generator is saved with the mids-object.
\end{Details}
\begin{Author}\relax
Stef van Buuren, Karin Oudshoorn, 2000
\end{Author}
\begin{References}\relax
Van Buuren, S. \& Oudshoorn, C.G.M. (2000). Multivariate Imputation by Chained Equations: 
MICE V1.0 User's manual. Report PG/VGZ/00.038, TNO Prevention and Health, Leiden.
\end{References}
\begin{SeeAlso}\relax
\end{SeeAlso}
\begin{Examples}
\begin{ExampleCode}
data(nhanes)
imp1 <- mice(nhanes,maxit=1)
imp2 <- mice.mids(imp1)

# yields the same result as
imp <- mice(nhanes,maxit=2)

# for example:
# 
# > imp$imp$bmi[1,]
#      1    2    3    4    5 
# 1 30.1 35.3 33.2 35.3 27.5
# > imp2$imp$bmi[1,]
#      1    2    3    4    5 
# 1 30.1 35.3 33.2 35.3 27.5
# 
\end{ExampleCode}
\end{Examples}

