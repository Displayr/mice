\HeaderA{mice.impute.mean}{Elementary Imputation Method: Simple Mean Imputation}{mice.impute.mean}
\keyword{misc}{mice.impute.mean}
\begin{Description}\relax
Imputes the arithmetic mean of the observed data
\end{Description}
\begin{Usage}
\begin{verbatim}
    mice.impute.mean(y, ry, x=NULL)
\end{verbatim}
\end{Usage}
\begin{Arguments}
\begin{ldescription}
\item[\code{y}] Incomplete data vector of length n
\item[\code{ry}] Vector of missing data pattern (FALSE=missing, TRUE=observed)
\item[\code{x}] Matrix (n x p) of complete covariates.

\end{ldescription}


\value{
A vector of length nmis with imputations.}
\end{Arguments}
\begin{Value}
A vector of length nmis with imputations.
\end{Value}
\begin{Section}{Warning}
Imputing the mean of a variable rarely produces appropriate inferences.
See Little and Rubin (1987).
\end{Section}
\begin{Author}\relax
Stef van Buuren, Karin Oudshoorn, 2000
\end{Author}
\begin{References}\relax
Van Buuren, S. \& Oudshoorn, C.G.M. (2000). Multivariate Imputation by Chained Equations: 
MICE V1.0 User's manual. Report PG/VGZ/00.038, TNO Prevention and Health, Leiden.

Little, R.J.A. and Rubin, D.B. (1987). Statistical Analysis with Missing Data. 
New York: John Wiley and Sons.
\end{References}
\begin{SeeAlso}\relax
\code{\LinkA{mice}{mice}}, \code{\LinkA{mean}{mean}}
\end{SeeAlso}

