\HeaderA{mice.impute.polyreg}{Elementary Imputation Method: Polytomous Regression}{mice.impute.polyreg}
\keyword{misc}{mice.impute.polyreg}
\begin{Description}\relax
Imputes missing data in a categorical variable using polytomous regression
\end{Description}
\begin{Usage}
\begin{verbatim}
mice.impute.polyreg(y, ry, x)
\end{verbatim}
\end{Usage}
\begin{Arguments}
\begin{ldescription}
\item[\code{y}] Incomplete data vector of length n
\item[\code{ry}] Vector of missing data pattern (FALSE=missing, TRUE=observed)
\item[\code{x}] Matrix (n x p) of complete covariates.
\end{ldescription}
\end{Arguments}
\begin{Details}\relax
Imputation for categorical response variables by the Bayesian 
polytomous regression model. See J.P.L. Brand (1999), Chapter 4,
Appendix B.

The method consists of the following steps:
\begin{enumerate}
\item Fit categorical response as a multinomial model 
\item Compute predicted categories
\item Add appropriate noise to predictions.
\end{enumerate}
This algorithm uses the function multinom from the libraries nnet and MASS
(Venables and Ripley).
\end{Details}
\begin{Value}
A vector of length nmis with imputations.
\end{Value}
\begin{Author}\relax
Stef van Buuren, Karin Oudshoorn, 2000
\end{Author}
\begin{References}\relax
Van Buuren, S. \& Oudshoorn, C.G.M. (2000). Multivariate Imputation by Chained Equations: 
MICE V1.0 User's manual. Report PG/VGZ/00.038, TNO Prevention and Health, Leiden.

Brand, J.P.L. (1999). Development, Implementation and Evaluation of Multiple Imputation Strategies for the Statistical Analysis of Incomplete Data Sets. Ph.D. Thesis, TNO Prevention and Health/Erasmus University Rotterdam. ISBN 90-74479-08-1.
\end{References}
\begin{SeeAlso}\relax
\code{\LinkA{mice}{mice}}, \code{\LinkA{multinom}{multinom}}
\end{SeeAlso}

