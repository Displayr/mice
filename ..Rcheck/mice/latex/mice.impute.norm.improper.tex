\HeaderA{mice.impute.norm.improper}{Elementary Imputation Method: Linear Regression Analysis (improper)}{mice.impute.norm.improper}
\keyword{misc}{mice.impute.norm.improper}
\begin{Description}\relax
Imputes univariate missing data using linear regression analysis (improper version)
\end{Description}
\begin{Usage}
\begin{verbatim}
    mice.impute.norm.improper(y, ry, x)
\end{verbatim}
\end{Usage}
\begin{Arguments}
\begin{ldescription}
\item[\code{y}] Incomplete data vector of length n
\item[\code{ry}] Vector of missing data pattern (FALSE=missing, TRUE=observed)
\item[\code{x}] Matrix (n x p) of complete covariates.
\end{ldescription}
\end{Arguments}
\begin{Details}\relax
This creates imputation using the spread around the fitted 
linear regression line of y given x, as fitted on the observed data.
\end{Details}
\begin{Value}
A vector of length nmis with imputations.
\end{Value}
\begin{Section}{Warning}
The function does not incorporate the variability of the regression
weights, so it is not 'proper' in the sense of Rubin. For small samples, 
variability of the mice.imputed data is therefore somewhat underestimated.
\end{Section}
\begin{Note}\relax
This function is provided mainly to allow comparison between proper
and improper norm methods.
\end{Note}
\begin{Author}\relax
Stef van Buuren, Karin Oudshoorn, 2000
\end{Author}
\begin{References}\relax
Van Buuren, S. \& Oudshoorn, C.G.M. (2000). Multivariate Imputation by Chained Equations: 
MICE V1.0 User's manual. Report PG/VGZ/00.038, TNO Prevention and Health, Leiden.

Brand, J.P.L. (1999). Development, Implementation and Evaluation of
Multiple Imputation Strategies for the Statistical Analysis of
Incomplete Data Sets. Ph.D. Thesis, TNO Prevention
and Health/Erasmus University Rotterdam. ISBN 90-74479-08-1.
\end{References}
\begin{SeeAlso}\relax
\code{\LinkA{mice}{mice}},  \code{\LinkA{mice.impute.norm}{mice.impute.norm}}
\end{SeeAlso}

