\HeaderA{mice.impute.sample}{Elementary Imputation Method: Simple Random Sample}{mice.impute.sample}
\keyword{misc}{mice.impute.sample}
\begin{Description}\relax
Imputes a random sample from the observed y data
\end{Description}
\begin{Usage}
\begin{verbatim}
mice.impute.sample(y, ry, x=NULL)
\end{verbatim}
\end{Usage}
\begin{Arguments}
\begin{ldescription}
\item[\code{y}] Incomplete data vector of length n
\item[\code{ry}] Vector of missing data pattern (FALSE=missing, TRUE=observed)
\item[\code{x}] Matrix (n x p) of complete covariates.
\end{ldescription}
\end{Arguments}
\begin{Details}\relax
This function takes a simple random sample from the observed values in
y, and returns these as imputations.
\end{Details}
\begin{Value}
A vector of length nmis with imputations.
\end{Value}
\begin{Author}\relax
Stef van Buuren, Karin Oudshoorn, 2000
\end{Author}
\begin{References}\relax
Van Buuren, S. \& Oudshoorn, C.G.M. (2000). Multivariate Imputation by Chained Equations: 
MICE V1.0 User's manual. Report PG/VGZ/00.038, TNO Prevention and Health, Leiden.
\end{References}

