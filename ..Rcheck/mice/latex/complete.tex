\HeaderA{complete}{Produces Imputed Flat Files
from Multiply Imputed Data Set (mids)}{complete}
\keyword{misc}{complete}
\begin{Description}\relax
Takes an object of type mids, fills in the missing data, and
returns the completed data in a specified format.
\end{Description}
\begin{Usage}
\begin{verbatim}
    complete(x, action=1)
\end{verbatim}
\end{Usage}
\begin{Arguments}
\begin{ldescription}
\item[\code{x}] An object of class \code{'mids'}
(created by the function \code{mice()}).
\item[\code{action}] If action is a scalar between 1 and \code{x\$m},
the function returns the data with the action's
imputation filled in. Thus, \code{action=1} returns
the first
completed data set. 
The can also be one of the following
strings: \code{"long"}, \code{"broad"}, \code{"repeated"}.
This has the following meaning:
\begin{description}
\item[\code{action="long"}] produces a long matrix with n*m
rows,containing all imputed data plus two additional
variables \code{"\_ID\_"} (containing the row.names)
and \code{"\_IMP\_"} (containing the imputation number).
\item[\code{action="broad"}] produces a broad matrix with m times
the number of columns in the original data.
The first ncol(\code{x\$data}) columns contain the first
imputed data matrix. Column names are changed to
reflect the imputation number.
\item[\code{action="repeated"}] produces a broad matrix with m times
\code{ncol(x\$data)} columns. The first m columns
give the filled-in first variable. Column names are
changed to reflect the imputation number.
\end{description}

\end{ldescription}
\end{Arguments}
\begin{Value}
A data frame with the imputed values filled in.
\end{Value}
\begin{Author}\relax
Stef van Buuren, Karin Oudshoorn, 2000
\end{Author}
\begin{References}\relax
Van Buuren, S. \& Oudshoorn, C.G.M. (2000). Multivariate Imputation by Chained Equations: 
MICE V1.0 User's manual. Report PG/VGZ/00.038, TNO Prevention and Health, Leiden.
\end{References}
\begin{SeeAlso}\relax
\code{\LinkA{mice}{mice}}, \code{\LinkA{mids}{mids}}
\end{SeeAlso}
\begin{Examples}
\begin{ExampleCode}
data(nhanes)
imp <- mice(nhanes)     # do default multiple imputation on a numeric matrix
mat <- complete(imp)    # fills in the first imputation
mat <- complete(imp, 3) # fills in the third imputation
mat <- complete(imp, "long") # produces a long matrix with stacked complete data
mat <- complete(imp, "b") # a broad matrix
cor(mat)                # for numeric mat, produces a blocked correlation matrix, where
            # each m*m block contains of the same variable pair over different
            # multiple imputations.
\end{ExampleCode}
\end{Examples}

