\documentclass{article}
\usepackage[ae,hyper]{Rd}
\begin{document}
\HeaderA{complete}{Produces Imputed Flat Files
from Multiply Imputed Data Set (mids)}{complete}
\keyword{misc}{complete}
\begin{Description}\relax
Takes an object of type mids, fills in the missing data, and
returns the completed data in a specified format.
\end{Description}
\begin{Usage}
\begin{verbatim}
    complete(x, action=1)
\end{verbatim}
\end{Usage}
\begin{Arguments}
\begin{ldescription}
\item[\code{x}] An object of class \code{'mids'}
(created by the function \code{mice()}).
\item[\code{action}] If action is a scalar between 1 and \code{x\$m},
the function returns the data with the action's
imputation filled in. Thus, \code{action=1} returns
the first
completed data set. 
The can also be one of the following
strings: \code{"long"}, \code{"broad"}, \code{"repeated"}.
This has the following meaning:
\begin{description}
\item[\code{action="long"}] produces a long matrix with n*m
rows,containing all imputed data plus two additional
variables \code{"\_ID\_"} (containing the row.names)
and \code{"\_IMP\_"} (containing the imputation number).
\item[\code{action="broad"}] produces a broad matrix with m times
the number of columns in the original data.
The first ncol(\code{x\$data}) columns contain the first
imputed data matrix. Column names are changed to
reflect the imputation number.
\item[\code{action="repeated"}] produces a broad matrix with m times
\code{ncol(x\$data)} columns. The first m columns
give the filled-in first variable. Column names are
changed to reflect the imputation number.
\end{description}

\end{ldescription}
\end{Arguments}
\begin{Value}
A data frame with the imputed values filled in.
\end{Value}
\begin{Author}\relax
Stef van Buuren, Karin Oudshoorn, 2000
\end{Author}
\begin{References}\relax
Van Buuren, S. \& Oudshoorn, C.G.M. (2000). Multivariate Imputation by Chained Equations: 
MICE V1.0 User's manual. Report PG/VGZ/00.038, TNO Prevention and Health, Leiden.
\end{References}
\begin{SeeAlso}\relax
\code{\LinkA{mice}{mice}}, \code{\LinkA{mids}{mids}}
\end{SeeAlso}
\begin{Examples}
\begin{ExampleCode}
data(nhanes)
imp <- mice(nhanes)     # do default multiple imputation on a numeric matrix
mat <- complete(imp)    # fills in the first imputation
mat <- complete(imp, 3) # fills in the third imputation
mat <- complete(imp, "long") # produces a long matrix with stacked complete data
mat <- complete(imp, "b") # a broad matrix
cor(mat)                # for numeric mat, produces a blocked correlation matrix, where
            # each m*m block contains of the same variable pair over different
            # multiple imputations.
\end{ExampleCode}
\end{Examples}

\HeaderA{glm.mids}{Generelized Linear Regression on Multiply Imputed Data}{glm.mids}
\keyword{misc}{glm.mids}
\begin{Description}\relax
Performs repeated glm on a multiply imputed data set
\end{Description}
\begin{Usage}
\begin{verbatim}
    glm.mids(formula, data, ...)
\end{verbatim}
\end{Usage}
\begin{Arguments}
\begin{ldescription}
\item[\code{formula}] a formula expression as for other regression models, of the form 
response ~ predictors. See the documentation
of \code{\LinkA{lm}{lm}} and \code{\LinkA{formula}{formula}} for details.
\item[\code{data}] An object of type \code{mids}, which stands for 'multiply imputed data set', typically
created by function \code{mice()}.
\item[\code{...}] Additional parameters passed to \code{\LinkA{glm}{glm}}.
\end{ldescription}
\end{Arguments}
\begin{Details}\relax
see \code{\LinkA{glm}{glm}}
\end{Details}
\begin{Value}
An objects of class \code{mira}, which stands for 'multiply imputed repeated analysis'.
This object contains m \code{glm.objects}, plus some descriptive information.
\end{Value}
\begin{Author}\relax
Stef van Buuren, Karin Oudshoorn, 2000
\end{Author}
\begin{References}\relax
Van Buuren, S. \& Oudshoorn, C.G.M. (2000). Multivariate Imputation by Chained Equations: 
MICE V1.0 User's manual. Report PG/VGZ/00.038, TNO Prevention and Health, Leiden.
\end{References}
\begin{SeeAlso}\relax
\code{\LinkA{glm}{glm}},  \code{\LinkA{mids}{mids}}, \code{\LinkA{mira}{mira}}
\end{SeeAlso}
\begin{Examples}
\begin{ExampleCode}
data(nhanes)
imp <- mice(nhanes)     # do default multiple imputation on a numeric matrix
glm.mids((hyp==2)~bmi+chl,data=imp)
    # fit
    # $call:
    # glm.mids(formula = (hyp == 2) ~ bmi + chl, data = imp)
    # 
    # $call1:
    # mice(data = nhanes)
    # 
    # $nmis:
    #  age bmi hyp chl 
    #    0   9   8  10
    # 
    # $analyses:
    # $analyses[[1]]:
    # Call:
    # glm(formula = formula, data = data.i)
    # 
    # Coefficients:
    #  (Intercept)         bmi         chl 
    #   -0.4746337 -0.01565534 0.005417846
    # 
    # Degrees of Freedom: 25 Total; 22 Residual
    # Residual Deviance: 2.323886 
    # 
    # $analyses[[2]]:
    # Call:
    # glm(formula = formula, data = data.i)
    # 
    # Coefficients:
    #  (Intercept)         bmi         chl 
    #   -0.1184695 -0.02885779 0.006090282
    # 
    # Degrees of Freedom: 25 Total; 22 Residual
    # Residual Deviance: 3.647927 
    # 
    # $analyses[[3]]:
    # Call:
    # glm(formula = formula, data = data.i)
    # 
    # Coefficients:
    #  (Intercept)          bmi         chl 
    #   -0.1503616 -0.003002851 0.002130091
    # 
    # Degrees of Freedom: 25 Total; 22 Residual
    # Residual Deviance: 3.799126 
    # 
    # $analyses[[4]]:
    # Call:
    # glm(formula = formula, data = data.i)
    # 
    # Coefficients:
    #  (Intercept)        bmi         chl 
    #  0.009442083 -0.0237619 0.004631881
    # 
    # Degrees of Freedom: 25 Total; 22 Residual
    # Residual Deviance: 3.874522 
    # 
    # $analyses[[5]]:
    # Call:
    # glm(formula = formula, data = data.i)
    # 
    # Coefficients:
    #  (Intercept)         bmi         chl 
    #   0.09932161 -0.02168292 0.003857599
    # 
    # Degrees of Freedom: 25 Total; 22 Residual
    # Residual Deviance: 4.025066 
    # 
    # 
    # > 
    # 

\end{ExampleCode}
\end{Examples}

\HeaderA{lm.mids}{Linear Regression on Multiply Imputed Data}{lm.mids}
\keyword{misc}{lm.mids}
\begin{Description}\relax
Performs repeated linear regression on multiply imputed data set
\end{Description}
\begin{Usage}
\begin{verbatim}
lm.mids(formula, data, ...)
\end{verbatim}
\end{Usage}
\begin{Arguments}
\begin{ldescription}
\item[\code{formula}] a formula object, with the response on the left of a ~ operator, and the 
terms, separated by + operators, on the right.
\item[\code{data}] An object of type 'mids', which stands for 'multiply imputed data set', typically
created by function \code{mice()}.
\item[\code{...}] Additional parameters passed to \code{\LinkA{lm}{lm}}
\end{ldescription}
\end{Arguments}
\begin{Value}
An objects of class 'mira', which stands for 'multiply imputed repeated analysis'.
This object contains m lm.objects, plus some descriptive information.
\end{Value}
\begin{Author}\relax
Stef van Buuren, Karin Oudshoorn, 2000
\end{Author}
\begin{References}\relax
Van Buuren, S. \& Oudshoorn, C.G.M. (2000). Multivariate Imputation by Chained Equations: 
MICE V1.0 User's manual. Report PG/VGZ/00.038, TNO Prevention and Health, Leiden.
\end{References}
\begin{SeeAlso}\relax
\code{\LinkA{lm}{lm}}, \code{\LinkA{mids}{mids}}, \code{\LinkA{mira}{mira}}
\end{SeeAlso}
\begin{Examples}
\begin{ExampleCode}
data(nhanes)
imp <- mice(nhanes)     # do default multiple imputation on a numeric matrix
fit <- lm.mids(bmi~hyp+chl,data=imp)
\end{ExampleCode}
\end{Examples}

\HeaderA{md.pattern}{Missing Data Pattern}{md.pattern}
\keyword{misc}{md.pattern}
\begin{Description}\relax
Display missing-data patterns.
\end{Description}
\begin{Usage}
\begin{verbatim}
md.pattern(x)
\end{verbatim}
\end{Usage}
\begin{Arguments}
\begin{ldescription}
\item[\code{x}] A data frame or a matrix containing the incomplete data. 
Missing values are coded as NA's. 
\end{ldescription}
\end{Arguments}
\begin{Details}\relax
This function is useful for investigating any structure of missing 
observation in the data. In specific case, the missing data pattern 
could be (nearly) monotone. Monotonicity can be used to simplify the 
imputation model. See Schafer (1997) for details. Also, the missing
pattern could suggest which variables could potentially be useful for
imputation of missing entries.
\end{Details}
\begin{Value}
A matrix with \code{ncol(x)+1} columns, in which each row corresponds to
a missing data pattern (1=observed, 0=missing). 
Rows and columns are sorted in increasing amounts of missing 
information. The last column and row contain row and column counts,
respectively.
\end{Value}
\begin{Author}\relax
Stef van Buuren, Karin Oudshoorn, 2000
\end{Author}
\begin{References}\relax
Schafer, J.L. (1997), Analysis of multivariate incomplete data. 
London: Chapman\&Hall.
\end{References}
\begin{Examples}
\begin{ExampleCode}
data(nhanes)
md.pattern(nhanes)
#     age hyp bmi chl    
#  13   1   1   1   1  0
#   1   1   1   0   1  1
#   3   1   1   1   0  1
#   1   1   0   0   1  2
#   7   1   0   0   0  3
#   0   8   9  10 27

\end{ExampleCode}
\end{Examples}

\HeaderA{mice.internal}{internal mice functions}{mice.internal}
\aliasA{.norm.draw}{mice.internal}{.norm.draw}
\aliasA{.pmm.match}{mice.internal}{.pmm.match}
\aliasA{check.imputationMethod}{mice.internal}{check.imputationMethod}
\aliasA{check.predictorMatrix}{mice.internal}{check.predictorMatrix}
\aliasA{check.visitSequence}{mice.internal}{check.visitSequence}
\aliasA{data.frame.to.matrix}{mice.internal}{data.frame.to.matrix}
\aliasA{is.mids}{mice.internal}{is.mids}
\aliasA{is.mipo}{mice.internal}{is.mipo}
\aliasA{is.mira}{mice.internal}{is.mira}
\aliasA{is.passive}{mice.internal}{is.passive}
\aliasA{logitreg}{mice.internal}{logitreg}
\aliasA{mipo}{mice.internal}{mipo}
\aliasA{padModel}{mice.internal}{padModel}
\aliasA{sampler}{mice.internal}{sampler}
\keyword{misc}{mice.internal}
\begin{Description}\relax
Internal functions for package mice.
\end{Description}
\begin{Usage}
\begin{verbatim}
check.imputationMethod(imputationMethod, defaultImputationMethod, visitSequence, data, nmis, nvar)
check.predictorMatrix(predictorMatrix, nmis, nvar)
check.visitSequence(visitSequence, nmis, nvar)
data.frame.to.matrix(x)
is.mids(data)
is.mipo(data)
is.mira(data)
is.passive(string)
logitreg(x, y, wt = rep(1, length(y)), intercept = TRUE, start, trace = TRUE, ...)
.norm.draw(y, ry, x)
padModel(data, imputationMethod, predictorMatrix, visitSequence, nmis, nvar)
.pmm.match(z, yhat = yhat, y = y)
sampler(p, data, m, imp, r, visitSequence, maxit, printFlag)
\end{verbatim}
\end{Usage}
\begin{Arguments}
\begin{ldescription}
\item[\code{imputationMethod, defaultImputationMethod, visitSequence, data, nmis, nvar,
predictorMatrix, x, y, ry, string, wt, intercept, start, trace, z, yhat, p, m,
imp, r, maxit, printFlag, ...}] arguments
\end{ldescription}
\end{Arguments}
\begin{Author}\relax
Stef van Buuren, Karin Oudshoorn, 2000
\end{Author}

\HeaderA{mice}{Multivariate Imputation by Chained Equations}{mice}
\keyword{misc}{mice}
\begin{Description}\relax
Produces an object of class "mids", which stands
for 'multiply imputed data set'.
\end{Description}
\begin{Usage}
\begin{verbatim}
mice(data, m = 5, 
    imputationMethod = vector("character",length=ncol(data)), 
    predictorMatrix = (1 - diag(1, ncol(data))),
    visitSequence = (1:ncol(data))[apply(is.na(data),2,any)], 
    defaultImputationMethod=c("pmm","logreg","polyreg"),
    maxit = 5, 
    diagnostics = TRUE, 
    printFlag = TRUE,
    seed = NA)
\end{verbatim}
\end{Usage}
\begin{Arguments}
\begin{ldescription}
\item[\code{data}] A data frame or a matrix containing the incomplete
data. Missing values are coded as NA's.
\item[\code{m}] Number of multiple imputations.
If omitted, m=5 is used.
\item[\code{imputationMethod}] Can be either a string,
or a vector of strings with length ncol(data),
specifying the elementary imputation method to be used
for each column in data. If specified as a single
string, the given method will be used for all columns.
The default imputation method (when no argument is specified)
depends on the measurement level of the target column and
are specified by the \code{defaultImputationMethod} argument.
Columns that need not be imputed have method \code{""}.
See details for more inromation.
\item[\code{predictorMatrix}] A square matrix of size \code{ncol(data)} containing 0/1 data specifying
the set of predictors to be used for each target column. Rows correspond
to target variables (i.e. variables to be imputed), in the sequence as
they appear in data. A value of '1' means that the column variable is
used as a predictor for the target variable (in the rows). The diagonal
of \code{predictorMatrix} must be zero. The default for \code{predictorMatrix} is that
all other columns are used as predictors (sometimes called massive
imputation).
\item[\code{visitSequence}] A vector of integers of arbitrary length, specifying the column indices
of the visiting sequence. The visiting sequence is the column order that
is used to impute the data during one iteration of the algorithm. A
column may be visited more than once. All incomplete columns that are
used as predictors should be visited, or else the function will stop
with an error. The default sequence 1:ncol(data) implies that columns
are imputed from left to right.
\item[\code{defaultImputationMethod}] A vector of three strings containing the default imputation methods for numerical columns, factor 
columns with 2 levels, and factor columns with more than two levels, respectively. If nothing is 
specified, the following defaults will be used:
\code{pmm}, predictive mean matching (numeric data);
\code{logreg}, logistic regression imputation (binary data, factor with 2 levels);
\code{polyreg}, polytomous regression imputation categorical data (factor >= 2 levels).
\item[\code{maxit}] A scalar giving the number of iterations. The default is 5.
\item[\code{diagnostics}] A Boolean flag. If \code{TRUE}, diagnostic
information will be appended to the value of the function. If
\code{FALSE}, only the imputed data are saved. The default is \code{TRUE}.
\item[\code{printFlag}] 
\item[\code{seed}] An integer between 0 and 1000 that is used by the
set.seed function for offsetting the random number generator. Default is to leave the random number generator alone.
\end{ldescription}
\end{Arguments}
\begin{Details}\relax
Generates multiple imputations for incomplete multivariate data by Gibbs
Sampling. Missing data can occur anywhere in the data. The algorithm
imputes an incomplete column (the target column) by generating
oappropriate imputation values given other columns in the data. Each
incomplete column must act as a target column, and has its own specific
set of predictors. The default predictor set consists of all other
columns in the data. For predictors that are incomplete themselves, the
most recently generated imputations are used to complete the predictors
prior to imputation of the target column. 

A separate univariate imputation model can be specified for each
column. The default imputation method depends on the measurement level
of the target column. In addition to these, several other methods are
provided. Users may also write their own imputation functions, and call
these from within the algorithm. 

In some cases, an imputation model may need transformed data in addition
to the original data (e.g. log or quadratic transforms). In order to
maintain consistency among different transformations of the same data,
the function has a special built-in method using the \code{\textasciitilde{}} mechanism. This
method can be used to ensure that a data transform always depends on the
most recently generated imputations in the untransformed (active)
column.  

The data may contain categorical variables that are used in a
regressions on other variables. The algorithm creates dummy variables
for the categories of these variables, and imputes these from the
corresponding categorical variable. 

Built-in imputation methods are:

\describe{
\item[norm] Bayesian linear regression (Numeric)
\item[pmm] Predictive mean matching (Numeric)   
\item[mean] Unconditional mean imputation (Numeric)
\item[logreg] Logistic regression (2 categories)        
\item[logreg2] Logistic regression (direct minimization)(2 categories)
\item[polyreg] Polytomous logistic regression (>= 2 categories)
\item[lda] Linear discriminant analysis (>= 2 categories)        
\item[sample] Random sample from the observed values (Any)
\item[Special method] If the first character of the elementary
method is a \code{\textasciitilde{}}, then the string is interpreted as the formula argument
in a call to \code{model.frame(formula, data[!r[,j],])}. This provides a simple
mechanism for specifying a large variety of dependencies among the
variables. For example transformed versions of imputed variables,
recodes, interactions, sum scores, and so on, that may themselves be
needed in other parts of the algoritm, can be specified in this
way. Note that the \code{\textasciitilde{}} mechanism works only on those entries which have
missing values in the target column. The user should make sure that the
combined observed and imputed parts of the target column make sense. One
easy way to create consistency is by coding all entries in the target as
\code{NA}, but for large data sets, this could be inefficient. Moreover, this
will not work in S-Plus 4.5. Though not strictly needed, it is often
useful to specify \code{visitSequence} such that the column that is imputed by
the \code{\textasciitilde{}} mechanism is visited each time after one of its predictors was
visited. In that way, deterministic relation between columns will always
be synchronized.
}

For example, for the j'th column, the \code{impute.norm} function that implements the 
Bayesian linear regression method can be called by specifying the string "norm" 
as the j'th entry in the vector of strings. 

The user can write his or her own imputation function, say
\code{impute.myfunc}, and call it for all columns by specifying
\code{imputationMethod="myfunc"}, or for specific columns by specifying
\code{imputationMethod=c("norm","myfunc",...)}.

\emph{side effects:}
Some elementary imputation method require access to the nnet or MASS
libraries of Venables \& Ripley. Where needed, these libraries will be
attached.
\end{Details}
\begin{Value}
An object of class mids, which stands for 'multiply imputed data set'. For 
a description of the object, see the documentation on \code{\LinkA{mids}{mids}}.
\end{Value}
\begin{Author}\relax
Stef van Buuren, Karin Oudshoorn, 2000
\end{Author}
\begin{References}\relax
Van Buuren, S. and Oudshoorn, C.G.M.. (1999). Flexible multivariate
imputation by MICE. Report PG/VGZ/99.054, TNO Prevention and Health,
Leiden. 

Van Buuren, S. \& Oudshoorn, C.G.M. (2000). Multivariate Imputation by
Chained Equations:  
MICE V1.0 User's manual. Report PG/VGZ/00.038, TNO Prevention and
Health, Leiden. 

Van Buuren, S., Boshuizen, H.C. and Knook, D.L. (1999). Multiple
imputation of missing blood pressure covariates in survival
analysis. Statistics in Medicine, 18, 681-694. 

Brand, J.P.L. (1999). Development, implementation and evaluation of multiple imputation strategies for the statistical analysis of incomplete data sets. Dissertation, TNO Prevention and Health, Leiden and Erasmus University, Rotterdam.
\end{References}
\begin{SeeAlso}\relax
\code{\LinkA{complete}{complete}}, \code{\LinkA{mids}{mids}}, \code{\LinkA{lm.mids}{lm.mids}}, \code{\LinkA{set.seed}{set.seed}}
\end{SeeAlso}
\begin{Examples}
\begin{ExampleCode}
data(nhanes)
imp <- mice(nhanes)     # do default multiple imputation on a numeric matrix
imp
imp$imputations$bmi     # and list the actual imputations 
complete(imp)       # show the first completed data matrix
lm.mids(chl~age+bmi+hyp, imp)   # repeated linear regression on imputed data

data(nhanes2)
mice(nhanes2,im=c("sample","pmm","logreg","norm")) # imputation on mixed data with a different method per column
\end{ExampleCode}
\end{Examples}

\HeaderA{mice.impute.lda}{Elementary Imputation Method: Linear Discriminant Analysis}{mice.impute.lda}
\keyword{misc}{mice.impute.lda}
\begin{Description}\relax
Imputes univariate missing data using linear discriminant analysis
\end{Description}
\begin{Usage}
\begin{verbatim}
mice.impute.lda(y, ry, x)
\end{verbatim}
\end{Usage}
\begin{Arguments}
\begin{ldescription}
\item[\code{y}] Incomplete data vector of length n
\item[\code{ry}] Vector of missing data pattern (FALSE=missing, TRUE=observed)
\item[\code{x}] Matrix (n x p) of complete covariates.
\end{ldescription}
\end{Arguments}
\begin{Details}\relax
Imputation of categorical response variables by linear discriminant
analysis. This function uses the Venables/Ripley functions
lda and predict.lda to compute posterior probabilities for
each incomplete case, and draws the imputations from this 
posterior.
\end{Details}
\begin{Value}
A vector of length nmis with imputations.
\end{Value}
\begin{Section}{Warning}
The function does not incorporate the variability of the discriminant 
weight, so it is not 'proper' in the sense of Rubin. For small samples
and rare categories in the y, variability of the mice.imputed data could 
therefore be somewhat underestimated.
\end{Section}
\begin{Note}\relax
This function can be called from within the Gibbs sampler by specifying 
'lda' in the imputationMethod argument. 
This method is usually faster and uses less resources than 
\code{\LinkA{mice.impute.polyreg}{mice.impute.polyreg}}.
\end{Note}
\begin{Author}\relax
Stef van Buuren, Karin Oudshoorn, 2000
\end{Author}
\begin{References}\relax
Van Buuren, S. \& Oudshoorn, C.G.M. (2000). Multivariate Imputation by Chained Equations: 
MICE V1.0 User's manual. Report PG/VGZ/00.038, TNO Prevention and Health, Leiden.

Brand, J.P.L. (1999). Development, Implementation and Evaluation of
Multiple Imputation Strategies for the Statistical Analysis of
Incomplete Data Sets.
Ph.D. Thesis, TNO Prevention and Health/Erasmus University Rotterdam. ISBN 90-74479-08-1. 

Venables, W.N. \& Ripley, B.D. (1999). Modern applied statistics with S-Plus (3rd ed). Springer, Berlin.
\end{References}
\begin{SeeAlso}\relax
\code{\LinkA{mice}{mice}}, \code{\LinkA{lda}{lda}}, \code{\LinkA{predict.lda}{predict.lda}}
\end{SeeAlso}

\HeaderA{mice.impute.logreg}{Elementary Imputation Method: Logistic Regression}{mice.impute.logreg}
\keyword{misc}{mice.impute.logreg}
\begin{Description}\relax
Imputes univariate missing data using logistic regression.
\end{Description}
\begin{Usage}
\begin{verbatim}
    mice.impute.logreg(y, ry, x)
\end{verbatim}
\end{Usage}
\begin{Arguments}
\begin{ldescription}
\item[\code{y}] Incomplete data vector of length n
\item[\code{ry}] Vector of missing data pattern of length n (FALSE=missing,
TRUE=observed) 
\item[\code{x}] Matrix (n x p) of complete covariates.
\end{ldescription}
\end{Arguments}
\begin{Details}\relax
Imputation for binary response variables by the Bayesian 
logistic regression model. See Rubin (1987, p. 169-170) for
a description of the method.
The method consists of the following steps:
\Enumerate{
\item Fit a logit, and find (bhat, V(bhat))
\item Draw BETA from N(bhat, V(bhat))
\item Compute predicted scores for m.d., i.e. logit-1(X BETA)
\item Compare the score to a random (0,1) deviate, and mice.impute.}
The method relies on the standard glm.fit function.
\end{Details}
\begin{Value}
\begin{ldescription}
\item[\code{imp}] A vector of length nmis with imputations (0 or 1).
\end{ldescription}
\end{Value}
\begin{Note}\relax
An alternative is mice.impute.logreg2.
\end{Note}
\begin{Author}\relax
Stef van Buuren, Karin Oudshoorn, 2000
\end{Author}
\begin{References}\relax
Van Buuren, S. \& Oudshoorn, C.G.M. (2000). Multivariate Imputation by Chained Equations: 
MICE V1.0 User's manual. Report PG/VGZ/00.038, TNO Prevention and Health, Leiden.

Brand, J.P.L. (1999). Development, Implementation and Evaluation of Multiple Imputation Strategies for 
the Statistical Analysis of Incomplete Data Sets. Ph.D. Thesis, 
TNO Prevention and Health/Erasmus University Rotterdam. ISBN 90-74479-08-1.
\end{References}
\begin{SeeAlso}\relax
\code{\LinkA{mice}{mice}}, \code{\LinkA{glm}{glm}}, \code{\LinkA{glm.fit}{glm.fit}},
\code{\LinkA{mice.impute.logreg2}{mice.impute.logreg2}}
\end{SeeAlso}

\HeaderA{mice.impute.logreg2}{Elementary Imputation Method: Logistic Regression}{mice.impute.logreg2}
\keyword{misc}{mice.impute.logreg2}
\begin{Description}\relax
Imputes univariate missing data using logistic regression.
\end{Description}
\begin{Usage}
\begin{verbatim}
    imp <- mice.impute.logreg2(y, ry, x)
\end{verbatim}
\end{Usage}
\begin{Arguments}
\begin{ldescription}
\item[\code{y}] Incomplete data vector of length n
\item[\code{ry}] Vector of missing data pattern of length n (FALSE=missing,
TRUE=observed)
\item[\code{x}] Matrix (n x p) of complete covariates.

\end{ldescription}
\end{Arguments}
\begin{Details}\relax
Imputation for binary response variables by the Bayesian 
logistic regression model. See Rubin (1987, p. 169-170) for
a description of the method.
The method consists of the following steps:
\Enumerate{
\item Fit a logit, and find (bhat, V(bhat))
\item Draw BETA from N(bhat, V(bhat))
\item Compute predicted scores for m.d., i.e. logit-1(X BETA)
\item Compare the score to a random (0,1) deviate, and mice.impute.
}
This method uses direct minimization of the likelihood function
by means of V\&R function logitreg (V\&R, 2nd ed, p. 293).
\end{Details}
\begin{Value}
\begin{ldescription}
\item[\code{imp}] A vector of length nmis with imputations (0 or 1).
\end{ldescription}
\end{Value}
\begin{Note}\relax
An alternative is mice.impute.logreg.
\end{Note}
\begin{Author}\relax
Stef van Buuren, Karin Oudshoorn, 2000
\end{Author}
\begin{References}\relax
Van Buuren, S. \& Oudshoorn, C.G.M. (2000). Multivariate Imputation by Chained Equations: 
MICE V1.0 User's manual. Report PG/VGZ/00.038, TNO Prevention and Health, Leiden.

Brand, J.P.L. (1999). Development, Implementation and Evaluation
of Multiple
Imputation Strategies for the Statistical Analysis of Incomplete
Data Sets.
Ph.D. Thesis, TNO Prevention and Health/Erasmus University Rotterdam. ISBN 90-74479-08-1. 

Venables, W.N. \& Ripley, B.D. (1997). Modern applied statistics with S-Plus (2nd ed). Springer, Berlin.
\end{References}
\begin{SeeAlso}\relax
\code{\LinkA{mice}{mice}}, \code{\LinkA{logitreg}{logitreg}}, \code{\LinkA{mice.impute.logreg}{mice.impute.logreg}}
\end{SeeAlso}

\HeaderA{mice.impute.mean}{Elementary Imputation Method: Simple Mean Imputation}{mice.impute.mean}
\keyword{misc}{mice.impute.mean}
\begin{Description}\relax
Imputes the arithmetic mean of the observed data
\end{Description}
\begin{Usage}
\begin{verbatim}
    mice.impute.mean(y, ry, x=NULL)
\end{verbatim}
\end{Usage}
\begin{Arguments}
\begin{ldescription}
\item[\code{y}] Incomplete data vector of length n
\item[\code{ry}] Vector of missing data pattern (FALSE=missing, TRUE=observed)
\item[\code{x}] Matrix (n x p) of complete covariates.

\end{ldescription}


\value{
A vector of length nmis with imputations.}
\end{Arguments}
\begin{Value}
A vector of length nmis with imputations.
\end{Value}
\begin{Section}{Warning}
Imputing the mean of a variable rarely produces appropriate inferences.
See Little and Rubin (1987).
\end{Section}
\begin{Author}\relax
Stef van Buuren, Karin Oudshoorn, 2000
\end{Author}
\begin{References}\relax
Van Buuren, S. \& Oudshoorn, C.G.M. (2000). Multivariate Imputation by Chained Equations: 
MICE V1.0 User's manual. Report PG/VGZ/00.038, TNO Prevention and Health, Leiden.

Little, R.J.A. and Rubin, D.B. (1987). Statistical Analysis with Missing Data. 
New York: John Wiley and Sons.
\end{References}
\begin{SeeAlso}\relax
\code{\LinkA{mice}{mice}}, \code{\LinkA{mean}{mean}}
\end{SeeAlso}

\HeaderA{mice.impute.norm}{Elementary Imputation Method: Linear Regression Analysis}{mice.impute.norm}
\keyword{misc}{mice.impute.norm}
\begin{Description}\relax
Imputes univariate missing data using linear regression analysis
\end{Description}
\begin{Usage}
\begin{verbatim}
mice.impute.norm(y, ry, x)
\end{verbatim}
\end{Usage}
\begin{Arguments}
\begin{ldescription}
\item[\code{y}] Incomplete data vector of length n
\item[\code{ry}] Vector of missing data pattern (FALSE=missing, TRUE=observed)
\item[\code{x}] Matrix (n x p) of complete covariates.
\end{ldescription}
\end{Arguments}
\begin{Details}\relax
Draws values of beta and sigma for Bayesian linear regression imputation 
of y given x according to Rubin p. 167.
\end{Details}
\begin{Value}
A vector of length nmis with imputations.
\end{Value}
\begin{Note}\relax
Using mice.impute.norm for all columns gives results similar to Schafer's norm 
method (Schafer, 1997), though much slower.
\end{Note}
\begin{Author}\relax
Stef van Buuren, Karin Oudshoorn, 2000
\end{Author}
\begin{References}\relax
Van Buuren, S. \& Oudshoorn, C.G.M. (2000). Multivariate Imputation by Chained Equations: 
MICE V1.0 User's manual. Report PG/VGZ/00.038, TNO Prevention and Health, Leiden.
Brand, J.P.L. (1999). Development, Implementation and Evaluation of Multiple Imputation Strategies for the Statistical Analysis of Incomplete Data Sets. Ph.D. Thesis, TNO Prevention and Health/Erasmus University Rotterdam. ISBN 90-74479-08-1. 

Schafer, J.L. (1997). Analysis of incomplete multivariate data. London: Chapman \& Hall.
\end{References}

\HeaderA{mice.impute.norm.improper}{Elementary Imputation Method: Linear Regression Analysis (improper)}{mice.impute.norm.improper}
\keyword{misc}{mice.impute.norm.improper}
\begin{Description}\relax
Imputes univariate missing data using linear regression analysis (improper version)
\end{Description}
\begin{Usage}
\begin{verbatim}
    mice.impute.norm.improper(y, ry, x)
\end{verbatim}
\end{Usage}
\begin{Arguments}
\begin{ldescription}
\item[\code{y}] Incomplete data vector of length n
\item[\code{ry}] Vector of missing data pattern (FALSE=missing, TRUE=observed)
\item[\code{x}] Matrix (n x p) of complete covariates.
\end{ldescription}
\end{Arguments}
\begin{Details}\relax
This creates imputation using the spread around the fitted 
linear regression line of y given x, as fitted on the observed data.
\end{Details}
\begin{Value}
A vector of length nmis with imputations.
\end{Value}
\begin{Section}{Warning}
The function does not incorporate the variability of the regression
weights, so it is not 'proper' in the sense of Rubin. For small samples, 
variability of the mice.imputed data is therefore somewhat underestimated.
\end{Section}
\begin{Note}\relax
This function is provided mainly to allow comparison between proper
and improper norm methods.
\end{Note}
\begin{Author}\relax
Stef van Buuren, Karin Oudshoorn, 2000
\end{Author}
\begin{References}\relax
Van Buuren, S. \& Oudshoorn, C.G.M. (2000). Multivariate Imputation by Chained Equations: 
MICE V1.0 User's manual. Report PG/VGZ/00.038, TNO Prevention and Health, Leiden.

Brand, J.P.L. (1999). Development, Implementation and Evaluation of
Multiple Imputation Strategies for the Statistical Analysis of
Incomplete Data Sets. Ph.D. Thesis, TNO Prevention
and Health/Erasmus University Rotterdam. ISBN 90-74479-08-1.
\end{References}
\begin{SeeAlso}\relax
\code{\LinkA{mice}{mice}},  \code{\LinkA{mice.impute.norm}{mice.impute.norm}}
\end{SeeAlso}

\HeaderA{mice.impute.passive}{Elementary Imputation Method: Passive Imputation}{mice.impute.passive}
\keyword{misc}{mice.impute.passive}
\begin{Description}\relax
Derive a new variable based on the mice.imputed data
\end{Description}
\begin{Usage}
\begin{verbatim}
mice.impute.passive(data, func)
\end{verbatim}
\end{Usage}
\begin{Arguments}
\begin{ldescription}
\item[\code{data}] A data frame
\item[\code{func}] A formula specifying the transformations on data
\end{ldescription}
\end{Arguments}
\begin{Details}\relax
This is a special imputation function for so-called passive imputation.
Using this function, the user can specify, at any point in the mice 
Gibbs sampling algorithm, a function on the (mice.imputed) data. 
This is useful, for example, to compute a cubic version
of a variable, a transformation like $Q=W/H^2$ based on two variables, 
or a mean variable like $(x1+x2+x3)/3$. The so derived variables might be
used in other places in the imputation model.
The function allows to dynamically derive virtually any function 
of the mice.imputed data at virtually any time.
\end{Details}
\begin{Value}
\begin{ldescription}
\item[\code{t}] The tranformed data.
\end{ldescription}
\end{Value}
\begin{Author}\relax
Stef van Buuren, Karin Oudshoorn, 2000
\end{Author}
\begin{References}\relax
Van Buuren, S. \& Oudshoorn, C.G.M. (2000). Multivariate Imputation by Chained Equations: 
MICE V1.0 User's manual. Report PG/VGZ/00.038, TNO Prevention and Health, Leiden.
\end{References}
\begin{SeeAlso}\relax
\code{\LinkA{mice}{mice}}
\end{SeeAlso}

\HeaderA{mice.impute.pmm}{Elementary Imputation Method: Linear Regression Analysis}{mice.impute.pmm}
\keyword{misc}{mice.impute.pmm}
\begin{Description}\relax
Imputes univariate missing data using predictive mean matching
\end{Description}
\begin{Usage}
\begin{verbatim}
mice.impute.pmm(y, ry, x)
\end{verbatim}
\end{Usage}
\begin{Arguments}
\begin{ldescription}
\item[\code{y}] Incomplete data vector of length n
\item[\code{ry}] Vector of missing data pattern (FALSE=missing, TRUE=observed)
\item[\code{x}] Matrix (n x p) of complete covariates.
\end{ldescription}
\end{Arguments}
\begin{Details}\relax
Imputation of y by predictive mean matching, based on
Rubin (p. 168, formulas a and b).
The procedure is as follows:
\begin{enumerate}
\item Draw beta and sigma from the proper posterior
\item Compute predicted values for yobs and ymis
\item For each ymis, find the observation with closest predicted
value, and take its observed y as the imputation.
\end{enumerate}
The matching is on yhat, NOT on y, which deviates from formula b.
\end{Details}
\begin{Value}
\begin{ldescription}
\item[\code{imp}] A vector of length nmis with imputations.
\end{ldescription}
\end{Value}
\begin{Author}\relax
Stef van Buuren, Karin Oudshoorn, 2000
\end{Author}
\begin{References}\relax
Van Buuren, S. \& Oudshoorn, C.G.M. (2000). Multivariate Imputation by Chained Equations: 
MICE V1.0 User's manual. Report PG/VGZ/00.038, TNO Prevention and Health, Leiden.

Rubin, D.B. (1987). Multiple imputation for nonresponse in surveys. New York: Wiley.
\end{References}

\HeaderA{mice.impute.polyreg}{Elementary Imputation Method: Polytomous Regression}{mice.impute.polyreg}
\keyword{misc}{mice.impute.polyreg}
\begin{Description}\relax
Imputes missing data in a categorical variable using polytomous regression
\end{Description}
\begin{Usage}
\begin{verbatim}
mice.impute.polyreg(y, ry, x)
\end{verbatim}
\end{Usage}
\begin{Arguments}
\begin{ldescription}
\item[\code{y}] Incomplete data vector of length n
\item[\code{ry}] Vector of missing data pattern (FALSE=missing, TRUE=observed)
\item[\code{x}] Matrix (n x p) of complete covariates.
\end{ldescription}
\end{Arguments}
\begin{Details}\relax
Imputation for categorical response variables by the Bayesian 
polytomous regression model. See J.P.L. Brand (1999), Chapter 4,
Appendix B.

The method consists of the following steps:
\begin{enumerate}
\item Fit categorical response as a multinomial model 
\item Compute predicted categories
\item Add appropriate noise to predictions.
\end{enumerate}
This algorithm uses the function multinom from the libraries nnet and MASS
(Venables and Ripley).
\end{Details}
\begin{Value}
A vector of length nmis with imputations.
\end{Value}
\begin{Author}\relax
Stef van Buuren, Karin Oudshoorn, 2000
\end{Author}
\begin{References}\relax
Van Buuren, S. \& Oudshoorn, C.G.M. (2000). Multivariate Imputation by Chained Equations: 
MICE V1.0 User's manual. Report PG/VGZ/00.038, TNO Prevention and Health, Leiden.

Brand, J.P.L. (1999). Development, Implementation and Evaluation of Multiple Imputation Strategies for the Statistical Analysis of Incomplete Data Sets. Ph.D. Thesis, TNO Prevention and Health/Erasmus University Rotterdam. ISBN 90-74479-08-1.
\end{References}
\begin{SeeAlso}\relax
\code{\LinkA{mice}{mice}}, \code{\LinkA{multinom}{multinom}}
\end{SeeAlso}

\HeaderA{mice.impute.sample}{Elementary Imputation Method: Simple Random Sample}{mice.impute.sample}
\keyword{misc}{mice.impute.sample}
\begin{Description}\relax
Imputes a random sample from the observed y data
\end{Description}
\begin{Usage}
\begin{verbatim}
mice.impute.sample(y, ry, x=NULL)
\end{verbatim}
\end{Usage}
\begin{Arguments}
\begin{ldescription}
\item[\code{y}] Incomplete data vector of length n
\item[\code{ry}] Vector of missing data pattern (FALSE=missing, TRUE=observed)
\item[\code{x}] Matrix (n x p) of complete covariates.
\end{ldescription}
\end{Arguments}
\begin{Details}\relax
This function takes a simple random sample from the observed values in
y, and returns these as imputations.
\end{Details}
\begin{Value}
A vector of length nmis with imputations.
\end{Value}
\begin{Author}\relax
Stef van Buuren, Karin Oudshoorn, 2000
\end{Author}
\begin{References}\relax
Van Buuren, S. \& Oudshoorn, C.G.M. (2000). Multivariate Imputation by Chained Equations: 
MICE V1.0 User's manual. Report PG/VGZ/00.038, TNO Prevention and Health, Leiden.
\end{References}

\HeaderA{mice.mids}{Multivariate Imputation by Chained Equations (Iteration Step)}{mice.mids}
\keyword{misc}{mice.mids}
\begin{Description}\relax
Takes a "mids"-object, and produces an new object of class "mids".
\end{Description}
\begin{Usage}
\begin{verbatim}
mice.mids(obj, maxit=1, diagnostics=TRUE, printFlag=TRUE)
\end{verbatim}
\end{Usage}
\begin{Arguments}
\begin{ldescription}
\item[\code{obj}] An object of class "mids", typically produces by a previous call
to \code{mice()} or \code{mice.mids()} 
\item[\code{maxit}] The number of additional Gibbs sampling iterations. 
\item[\code{diagnostics}] A Boolean flag. If TRUE, diagnostic information will be appended to 
the value of the function. If FALSE, only the imputed data are saved. 
The default is TRUE.
\item[\code{printFlag}] A Boolean flag. If TRUE, diagnostic information during the Gibbs sampling
iterations will be written to the command window.  The default is TRUE.
\end{ldescription}
\end{Arguments}
\begin{Details}\relax
This function enables the user to split up the computations of the 
Gibbs sampler into smaller parts. This is useful for the following
reasons:
\begin{itemize}
\item RAM memory may become easily exhausted if the number of iterations is 
large. Returning to prompt/session level may alleviate these problems.
\item The user can compute customized convergence statistics at specific
points, e.g. after each iteration, for monitoring convergence.
- For computing a 'few extra iterations'.
\end{itemize}
Note: The imputation model itself is specified in the mice() function
and cannot be changed with mice.mids.
The state of the random generator is saved with the mids-object.
\end{Details}
\begin{Author}\relax
Stef van Buuren, Karin Oudshoorn, 2000
\end{Author}
\begin{References}\relax
Van Buuren, S. \& Oudshoorn, C.G.M. (2000). Multivariate Imputation by Chained Equations: 
MICE V1.0 User's manual. Report PG/VGZ/00.038, TNO Prevention and Health, Leiden.
\end{References}
\begin{SeeAlso}\relax
\end{SeeAlso}
\begin{Examples}
\begin{ExampleCode}
data(nhanes)
imp1 <- mice(nhanes,maxit=1)
imp2 <- mice.mids(imp1)

# yields the same result as
imp <- mice(nhanes,maxit=2)

# for example:
# 
# > imp$imp$bmi[1,]
#      1    2    3    4    5 
# 1 30.1 35.3 33.2 35.3 27.5
# > imp2$imp$bmi[1,]
#      1    2    3    4    5 
# 1 30.1 35.3 33.2 35.3 27.5
# 
\end{ExampleCode}
\end{Examples}

\HeaderA{mids}{Multiply Imputed Data Set}{mids}
\aliasA{plot.mids}{mids}{plot.mids}
\aliasA{print.mids}{mids}{print.mids}
\aliasA{summary.mids}{mids}{summary.mids}
\keyword{misc}{mids}
\begin{Description}\relax
An object containing a multiply imputed data set. The
"mids" object is generated by the mice and mice.mids functions. The
"mids" class of objects has methods for the following generic
functions: \code{print}, \code{summary}, \code{plot}
\end{Description}
\begin{Usage}
\begin{verbatim}
## S3 method for class 'mids':
print(x, ...)
## S3 method for class 'mids':
summary(object, ...)
## S3 method for class 'mids':
plot(x, y, ...)
\end{verbatim}
\end{Usage}
\begin{Arguments}
\begin{ldescription}
\item[\code{x}] A mids object.
\item[\code{object}] A mids object.
\item[\code{y}] Not used.
\item[\code{...}] Not used.
\end{ldescription}
\end{Arguments}
\begin{Value}
\begin{ldescription}
\item[\code{call}] The call that created the object.
\item[\code{data}] A copy of the incomplete data set.
\item[\code{m}] The number of imputations.
\item[\code{nmis}] An array containing the number of missing observations per column.
\item[\code{imp}] A list of nvar components with the generated multiple imputations. 
Each part of the list is a \code{nmis[j]} by m matrix of imputed values for 
variable j.
\item[\code{imputationMethod}] A vector of strings of length(nvar) specifying the elementary 
imputation method per column.
\item[\code{predictorMatrix}] A square matrix of size \code{ncol(data)} containing 0/1 data specifying 
the predictor set.
\item[\code{visitSequence}] The sequence in which columns are visited.
\item[\code{seed}] The seed value of the solution.
\item[\code{iteration}] Last Gibbs sampling iteration number.
\item[\code{lastSeedValue}] The most recent seed value.
\item[\code{chainMean}] A list of m components. Each component is a \code{length(visitSequence)}
by maxit matrix containing the mean of the generated multiple 
imputations. The array can be used for monitoring convergence.
Note that observed data are not present in this mean.
\item[\code{chainCov}] A list with similar structure of itermean, containing the covariances 
of the imputed values.
\item[\code{pad}] A list containing various settings of the padded imputation model, 
i.e. the imputation model after creating dummy variables. Normally, 
this array is only useful for error checking.
\end{ldescription}
\end{Value}
\begin{Author}\relax
Stef van Buuren, Karin Oudshoorn, 2000
\end{Author}
\begin{References}\relax
Van Buuren, S. \& Oudshoorn, C.G.M. (2000). Multivariate Imputation by Chained Equations: 
MICE V1.0 User's manual. Report PG/VGZ/00.038, TNO Prevention and Health, Leiden.
\end{References}

\HeaderA{mipo}{Multiply Imputed Pooled Analysis}{mipo}
\aliasA{print.mipo}{mipo}{print.mipo}
\aliasA{summary.mipo}{mipo}{summary.mipo}
\keyword{misc}{mipo}
\begin{Description}\relax
The "mipo" object is generated by the \code{\LinkA{lm.mids}{lm.mids}} and \code{\LinkA{glm.mids}{glm.mids}} functions.
The "mipo" class of objects has methods for the following generic functions:
print, summary.
\end{Description}
\begin{Usage}
\begin{verbatim}
print.mipo(x,...)
summary.mipo(object,...)
\end{verbatim}
\end{Usage}
\begin{Arguments}
\begin{ldescription}
\item[\code{x, object}] An object containing the m fit objects of a complete data analysis, 
plus some additional information.
\item[\code{...}] not used.
\end{ldescription}
\end{Arguments}
\begin{Value}
\begin{ldescription}
\item[\code{call}] The call that created the mipo object.
\item[\code{call1}] The call that created the mira object that was used in 'call'.
\item[\code{call2}] The call that created the mids object that was used in 'call1'.
\item[\code{nmis}] An array containing the number of missing observations per column.
\item[\code{m}] Number of multiple imputations.
\item[\code{qhat}] An m x \code{npar} matrix containing the complete data estimates for the \code{npar} paremeters of the m complete data analyses.
\item[\code{u}] An m x \code{npar} x \code{npar} array containing the variance-covariance matrices of the m complete data analyses.
\item[\code{qbar}] The average of complete data estimates.
\item[\code{ubar}] The average of the variance-covariance matrix of the complete data estimes.
\item[\code{b}] The between imputation variance-covariance matrix.
\item[\code{t}] The total variance-covariance matrix.
\item[\code{r}] Relative increases in variance due to missing data
\item[\code{df}] Degrees of freedom associated with the t-statistics.
\item[\code{f}] Fractions of missing information.
\end{ldescription}
\end{Value}
\begin{Author}\relax
Stef van Buuren, Karin Oudshoorn, 2000
\end{Author}
\begin{References}\relax
Van Buuren, S. \& Oudshoorn, C.G.M. (2000). Multivariate Imputation by Chained Equations: 
MICE V1.0 User's manual. Report PG/VGZ/00.038, TNO Prevention and Health, Leiden.
\end{References}

\HeaderA{mira}{Multiply Imputed Repeated Analysis}{mira}
\aliasA{print.mira}{mira}{print.mira}
\aliasA{summary.mira}{mira}{summary.mira}
\keyword{misc}{mira}
\begin{Description}\relax
The "mira" object is generated by the lm.mids and glm.mids functions.
The "mira" class of objects has methods for the following generic functions:
print, summary.
\end{Description}
\begin{Usage}
\begin{verbatim}
print.mira(x,...)
summary.mira(object, correlation, ...)
\end{verbatim}
\end{Usage}
\begin{Arguments}
\begin{ldescription}
\item[\code{x, object}] An object containing the m fit objects of a complete data analysis, 
plus some additional information.
\item[\code{correlation}] 
\item[\code{...}] not used
\end{ldescription}
\end{Arguments}
\begin{Value}
\begin{ldescription}
\item[\code{call}] The call that created the object.
\item[\code{call1}] The call that created the mids object that was used in 'call'.
\item[\code{nmis}] An array containing the number of missing observations per column.
\item[\code{analyses}] A list of m components containing the individual fit objects from each of the m complete data analyses.
\end{ldescription}
\end{Value}
\begin{Author}\relax
Stef van Buuren, Karin Oudshoorn, 2000
\end{Author}
\begin{References}\relax
Van Buuren, S. \& Oudshoorn, C.G.M. (2000). Multivariate Imputation by Chained Equations: 
MICE V1.0 User's manual. Report PG/VGZ/00.038, TNO Prevention and Health, Leiden.
\end{References}

\HeaderA{nhanes}{nhanes data set}{nhanes}
\keyword{datasets}{nhanes}
\begin{Description}\relax
\end{Description}
\begin{Usage}
\begin{verbatim}data(nhanes)\end{verbatim}
\end{Usage}
\begin{Format}\relax
A data frame with 25 observations on the following 4 variables.
\describe{
\item[age] a numeric vector
\item[bmi] a numeric vector
\item[hyp] a numeric vector
\item[chl] a numeric vector
}
\end{Format}
\begin{Source}\relax
\end{Source}

\HeaderA{nhanes2}{nhanes2 data set}{nhanes2}
\keyword{datasets}{nhanes2}
\begin{Description}\relax
\end{Description}
\begin{Usage}
\begin{verbatim}data(nhanes2)\end{verbatim}
\end{Usage}
\begin{Format}\relax
A data frame with 25 observations on the following 4 variables.
\describe{
\item[age] a factor with levels \code{1} \code{2} \code{3}
\item[bmi] a numeric vector
\item[hyp] a factor with levels \code{1} \code{2}
\item[chl] a numeric vector
}
\end{Format}
\begin{Source}\relax
\end{Source}

\HeaderA{pool}{Multiple Imputation Pooling}{pool}
\keyword{misc}{pool}
\begin{Description}\relax
Pools the results of m repeated complete data analysis
\end{Description}
\begin{Usage}
\begin{verbatim}
pool(object, method="smallsample")
\end{verbatim}
\end{Usage}
\begin{Arguments}
\begin{ldescription}
\item[\code{object}] An object of class 'mira', produced by functions like lm.mids or glm.mids.
\item[\code{method}] A string describing the method to compute the degrees of freedom. 
The default value is "smallsample", which specifies the is 
Barnard-Rubin adjusted degrees of freedom (Barnard\& Rubin, 1999) 
for small samples. Specifying a different string 
produces the conventional degrees of freedom as in Rubin (1987).
\end{ldescription}
\end{Arguments}
\begin{Details}\relax
The function averages the estimates of the complete data model, 
computes the total variance over the repeated analyses, and computes
the relative increase in variance due to nonresponse and the fraction 
of missing information. The function relies on the availability
of
\Enumerate{
\item the estimates of the model, typically present as 'coefficients' in 
the fit object
\item an appropriate estimate of the variance-covariance matrix of the 
estimates per analyses.
}
R-Specific: The original use of Varcov has been removed to vcov (VR MASS).
\end{Details}
\begin{Value}
An object of class 'mipo', which stands for 'multiple imputation pooled'.
\end{Value}
\begin{Author}\relax
Stef van Buuren, Karin Oudshoorn, 2000
\end{Author}
\begin{References}\relax
Barnard, J. and Rubin, D.B. (1999). Small sample degrees of freedom with
multiple imputation. Biometrika, 86, 948-955.

Rubin, D.B. (1987). Multiple Imputation for Nonresponse in Surveys. 
New York: John Wiley and Sons.

Alzola, C.F. and Harrell, F.E. (1999). An introduction to S-Plus and the Hmisc 
and Design Libraries. http://hesweb1.med.virginia.edu/biostat/s/index.html.
\end{References}
\begin{SeeAlso}\relax
\code{\LinkA{lm.mids}{lm.mids}}, \code{\LinkA{glm.mids}{glm.mids}}, \code{\LinkA{vcov}{vcov}},
\code{\LinkA{print.mira}{print.mira}}, \code{\LinkA{summary.mira}{summary.mira}}
\end{SeeAlso}
\begin{Examples}
\begin{ExampleCode}
data(nhanes)
imp <- mice(nhanes)
fit <- lm.mids(bmi~hyp+chl,data=imp)
pool(fit)
#  Call: pool(object = fit)
#  Pooled coefficients:
#   (Intercept)       hyp        chl 
#      21.29782 -1.751721 0.04085703
#
#  Fraction of information about the coefficients missing due to nonrespons
#  e: 
#   (Intercept)       hyp       chl 
#     0.1592247 0.1738868 0.3117452
#
#  > summary(pool(fit))
#           est         se          t       df     Pr(>|t|) 
#  (Intercept)  21.29781702 4.33668150  4.9110863 16.95890 0.0001329371
#      hyp  -1.75172102 2.30620984 -0.7595671 16.39701 0.4582953905
#      chl   0.04085703 0.02532914  1.6130442 11.50642 0.1338044664
#             lo 95      hi 95 missing       fmi 
#  (Intercept)  12.14652927 30.4491048      NA 0.1592247
#      hyp  -6.63106456  3.1276225       8 0.1738868
#    chl  -0.01459414  0.0963082      10 0.3117452 
\end{ExampleCode}
\end{Examples}

\end{document}
